\documentclass[10pt]{beamer}

\usetheme[progressbar=frametitle]{metropolis}
\usepackage[italian]{babel}

\usepackage{appendixnumberbeamer}

\usepackage{booktabs}
\usepackage[scale=2]{ccicons}

\usepackage{pgfplots}
\usepgfplotslibrary{dateplot}

\usepackage{xspace}
\newcommand{\themename}{\textbf{\textsc{metropolis}}\xspace}

\title{TSP con pick up and delivery}
\subtitle{Progetto del corso di Ricerca Operativa}
\date{13 dicembre 2022}
\author{Michele Vaccari - Matricola 121955}
\institute{Università degli studi di Ferrara\\Corso di laurea magistrale in Ingegneria Informatica e dell'Automazione\\AA 2021-2022}

% logo of my university
\titlegraphic{%
  \begin{picture}(0,0)
    \put(305,0){\makebox(0,0)[rt]{\includegraphics[width=4cm]{../images/logo-unife}}}
  \end{picture}}

\begin{document}

\maketitle

\begin{frame}{Indice}
  \setbeamertemplate{section in toc}[sections numbered]
  \tableofcontents[hideallsubsections]
\end{frame}

\section{Introduzione}

\begin{frame}{Descrizione del problema}
	A partire dalla base (nodo \emph{0} del grafo) un corriere deve soddisfare \emph{n} richieste di prelievo e consegna di documenti:
	\begin{itemize}
		\item 
		ogni documento è prelevato in un nodo e consegnato in un altro nodo;
		\item
		ogni nodo è riferito a una singola richiesta, ma nel tragitto tra punto di prelievo e consegna si possono prelevare/consegnare altri documenti.
	\end{itemize}
	Noto il tempo di percorrenza dei singoli archi, si vuole minimizzare la durata del percorso, con partenza e rientro al deposito.
\end{frame}

%\begin{frame}[allowframebreaks]{Modello matematico}
%\end{frame}

%\begin{frame}[allowframebreaks]{Interpretazione dei risultati del modello teorico nel problema reale}
%\end{frame}

\begin{frame}[standout]
Grazie per l'attenzione
\end{frame}

\end{document}